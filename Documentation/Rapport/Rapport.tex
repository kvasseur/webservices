\documentclass{report}
\usepackage[latin1]{inputenc}
\usepackage{hyperref}

\title{%
    \begin{minipage}\linewidth
        \centering
        Rapport de projet\\
        \large WEB API REST de gestion d'une mediath\`{e}que
    \end{minipage}
}

\author{
	\textbf{K\'{e}vin VASSEUR} \\
	\'{E}tudiant M1 ISIDis - Universit\'{e} du Littoral C\^{o}te d'Opale\\
	vasseur.isn@gmail.com\\
	\\
	\textbf{Marie-Laure FASQUEL} \\
	\'{E}tudiante M1 ISIDis - Universit\'{e} du Littoral C\^{o}te d'Opale\\
	fasquel.ml@gmail.com\\
}
\date{\textbf{\today}}


\begin{document}
	\maketitle
	\begin{abstract}
		Ce projet est de cr\'{e}er une WEB API REST qui g\`{e}re une mediath\`{e}que. Ce projet prend lieu lors des s\'{e}ances de TP de Web Services. La date de lancement de ce projet est dat\'{e} au 13 mars 2017 et sa date de livraison est fix\'{e}e au samedi 1 avril 2017. Nous avons libre choix dans les technologies. \\
		
	\end{abstract}
	
	\tableofcontents	
	
	\chapter{Background}
	% ########## TECHNOS ##########
	\section{Technologies utilis\'{e}es}
		\subsection{Angular}
		\textbf{Version :}  \\
		
		\subsection{Cake PHP}
		\textbf{Version :} 3.4.3 \\
		
		\subsection{Bootstrap}
		\textbf{Version :}  \\		
		
	\section{Niveau des membres}
		...
		
	\section{Liens}
		\href{https://github.com/kvasseur/webservices}{Github} \\
		\href{https://trello.com/b/A1vuuQZb/webservices-api-rest-mediatheque}{Trello}
	
	
	% ########## DOCUMENTATION ##########	
	\chapter{Manuel d'installation}
	
	\chapter{Manuel d'utilisation}
		
	% ########## LIENS ##########
	
	
	
\end{document}